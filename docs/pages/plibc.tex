This section of the manual describes the minimalistic C library functions that were
created for EFSL. Since EFSL was disigned for ultimate portability, no assumptions about the
workings or even the presence of a C library could be made. Fortunately only very few functions
had to be created that mimicked the operations of well known C library functions.
\\
\begin{longtable}{|p{0.35\textwidth}|p{0.65\textwidth}|}
	
	\hline
	\multicolumn{2}{|c|}{
		\textbf{PLibC Functions}
	} \\
	\multicolumn{2}{|c|}{} \\
	\hline
	\hline
	\endfirsthead
	
	\hline
	\multicolumn{2}{|c|}{\textbf{PLibC Functions (continued)}} \\
	\hline
	\endhead

	\hline
	\endfoot
	
	\hline 
	\endlastfoot

	\code{strMatch} & \code{euint16 strMatch(eint8* bufa, eint8*bufb,euint32 n)} \\
	\hline
	\multicolumn{2}{|p{\textwidth}|}{
	This function compares the strings \code{bufa} and \code{bufb} for \code{n} bytes.
	It will return the number of bytes in that section that does not match. So if you
	want to compare two strings the return value should be 0, for the strings to match over
	the entire \code{n} area.
	}\\
	\hline
	
	\code{memCpy} & \code{void memCpy(void* psrc, void* pdest, euint32 size)} \\
	\hline
	\multicolumn{2}{|p{\textwidth}|}{
	This function will copy the contents at location \code{psrc} to location \code{pdest} over
	a range of \code{size} bytes.
	}\\
	\hline

	\code{memClr} & \code{void memClr(void *pdest,euint32 size)} \\
	\hline
	\multicolumn{2}{|p{\textwidth}|}{
	This function will set the memory at \code{pdest} to value \code{0x00} for a range of
	\code{size} bytes.
	}\\
	\hline

	\code{memSet} & \code{void memSet(void *pdest,euint32 size,euint8 data)} \\
	\hline
	\multicolumn{2}{|p{\textwidth}|}{
	This function will set the memory at \code{pdest} to value \code{data} for a range of
	\code{size} bytes.
	}\\
	\hline
\end{longtable}

